\chapter{Introduction}

% Define problem
% What our intentions are and briefly on what we did
% Say outline of report

In this report, we go through the process of completing the project of the course \textit{DATA11002 - Introduction to Machine Learning}. The tasks consisted of performing multiclass and binary classification on the New Particle Formation (NPF) dataset \cite{npf_paper}. First, we start by discussing the task at hand, including some background on it. The report structure is given at the end of this introduction.

Before we begin, we would like to state that all our code is available on github. Furthermore, random seeds were used throughout the code in order to preserve the reproducibility of the results. The \textit{scikit-learn} \cite{sklearn} library was used extensively especially for building our models.

\section{The Task}

NPF is an event where small particles about two to three nanometers in diameter start to form into larger new particles which are more than 100 nanometers in diameter. It takes place in the troposphere and is an important contributor to cloud formation \cite{npf_paper}. The basic characteristics of an NPF burst are particle growth rate and formation rate.

In our data, all days are classified into four categories. The first category is named \textit{nonevent} and contains days on which no NPF was noticed. The other three categories indicate that NPF took place and are categorized into two classes. Class I includes the days on which NPF was determined with a good confidence level, and class II includes days during which growth rate or formation rate accuracy is not as high. The class I events are further split into two groups. Group Ia includes very clear and strong particle formation events and group Ib contains the rest of class I events \cite{npf_paper}.

Our task focuses on predicting which of the four aforementioned NPF event types occurred using measurements of relative humidity, condensation sink, among other variables. From previous research on NPF, we know that some factors influence NPF more than others. For example, according to \cite{npf_paper_3}, relative humidity and condensation sink are capable of explaining 88\% of the new particle formation events. It was also found that nucleation occurs only with low relative humidity and condensation sink values.

\section{Report Structure}

The report is structured as follows. We will begin by exploring our data. Some choices and requirements for the models will be stated already during this first step. Next, we will introduce the models and explain why these models are suitable for our task. The pipeline to train these models is then explained. Lastly, we will share our results on the training and testing splits and present our conclusions. The self-grading report is given in the appendices.
